\documentclass{article}

\usepackage[margin=0.9in]{geometry}

% \usepackage{hyperref}

\usepackage{cite}
\usepackage{url}
\usepackage{graphicx}
\usepackage{color}

\usepackage{graphicx}

\usepackage{amssymb,amsmath}

\usepackage{parskip}

\usepackage{algorithm}

\usepackage[noend]{algorithmic}

\renewcommand{\thefootnote}{\fnsymbol{footnote}}

\newcommand{\fig}[1]{Fig.~\ref{fig:#1}}
\newcommand{\Name}{\emph{Hidden Target}~}

\newcommand{\Acronym}[1]{\ensuremath{{{\texttt{#1}}}}}
\newcommand{\Symbol}[1]{\ensuremath{\mathcal{#1}}}
\newcommand{\Function}[1]{\ensuremath{{\textsc{#1}}}}
\newcommand{\Constant}[1]{\ensuremath{{\texttt{#1}}}}
\newcommand{\Var}[1]{\ensuremath{{{\textsl{#1}}}}}
\newcommand{\False}{\Constant{false}}
\newcommand{\True}{\Constant{true}}
\newcommand{\Null}{\Constant{null}}
\newcommand{\Revision}[1]{\textcolor{red}{#1}}
\newcommand{\R}{\ensuremath{\mathbb{R}}}
\newcommand{\Traj}{\ensuremath{\zeta}}
\newcommand{\Tree}{\Symbol{T}}
\newcommand{\pair}[1]{\ensuremath{\langle#1\rangle}}


\begin{document}

\setlength{\parskip}{4pt} % 1ex plus 0.5ex minus 0.2ex}

\title{Personal Statement}

\author{Alexander J. Wallar \\ University of St Andrews, UK
\\ \url{http://aw204.host.cs.st-andrews.ac.uk}}

\maketitle

\section{Experience}

I have been active in research since high school and have published 4
first-author papers in robotics. I participated in the FIRST Tech Challenge and
FIRST Robotics Competition during high school. I was accepted as a high school
research assistant at the Computational Robotics Laboratory at the Catholic
University of America (CUA) following my high school graduation in June 2012.
At CUA, I developed interfaces for controlling the iRobot Create that included
moving the robot using hand gestures interpreted using the Microsoft Kinect,
and with an Android application that could interpret natural spoken language.
After my first year of university in 2013, I participated in a National Science
Foundation Research Experience for Undergraduates (REU) site at the University
of Notre Dame. At Notre Dame, I developed a computer vision library for
JavaScript that can be used to determine where a user is looking on the screen.
For this project, I received the Best Poster Prize for the REU site.

Once I returned to university after the summer, I became a research assistant
in the School of Psychology where I configured a novel experiment that involved
three active-shutter 3D displays of different sizes that can be viewed
simultaneously through beam splitters. The goal of this set-up was to determine
how people perceive 3D imagery. In parallel with working for the School of
Psychology, I worked as a research assistant for the School of Computer Science
developing computer vision algorithms that can translate monocular images into
series of impulses that can be relayed to a haptic interface. The goal of this
software was to allow people with sight impairments to perceive the real-world
using haptic feedback.

During the summer of 2014, I was part of the Naval Research Enterprise
Internship Program at the Naval Center for Applied Research in Artificial
Intelligence at the Naval Research Laboratory in Washington DC. I worked in the
Distributed Autonomous Systems group developing motion planning algorithms that
enable a group of unmanned aerial vehicles to provide persistent surveillance
of a given area. Each robot maximizes the quality of the sensory information
being collected whilst minimizing the risk of damage to the vehicle and the
risk of being detected by a hostile target on the ground. I have continued to
be affiliated with the Computational Robotics Laboratory at the Catholic
University of America as an undergraduate research assistant developing path
planning algorithms that enable swarms of robots to go from an initial
configuration to a goal configuration in highly dense dynamic environments.
Most recently, I have been working as student contractor for the Naval Center
for Applied Research in Artificial Intelligence at the Naval Research
Laboratory developing algorithms that allow swarms to accomplish several high
level tasks given topological maps of the environment. The experience I have
gained through my participation in research during my studies has prepared me
well for a PhD; it has taught me how to solve difficult problems with little
supervision, to be independent and self motivated, and how to present my work
to the scientific community.

I would like to attend the University of Bristol to solve problems whose
solutions can have a major impact for good in the world. Working with Dr.
Hauert will allow me to participate in cutting edge research in swarming
nanosystems for cancer applications. I also look forward to working at the
Bristol Robotics Laboratory, which is one of the best robotics laboratories in
Europe. My previous research experience in swarm robotics and emergent
behaviour provide me with the needed background to take on challenges in swarm
engineering.


\section{Project Aims}

Under the supervision of Dr. Hauert, I will be working on the control of large
robot swarms by dynamically modifying the environment.

Swarm behaviours arise from the interaction of robots and their local
environment. Each robot is typically given a controller that determines its
individual behaviour. The challenge is to design the robot controllers given a
desired emergent behaviour.

As swarms become larger in number and smaller in physical size, our ability to
control individuals in a reliable manner diminishes. Nanobots for cancer
applications for example work in the trillions and are only able to diffuse and
react to their environment~\cite{nano}. Instead of designing controllers for
individual robots, we aim to change their environment dynamically as a way to
control the swarm.

This could enable very large numbers of minimal agents to perform controlled
swarm behaviours through the actuation of limited elements in their
environment.

There is no obvious relation between the design of the environment and the
emergent behaviour of a swarm. The challenge is to understand how the
environment should actively be controlled to achieve a desired swarm behaviour.
We hope this will result in a paradigm shift in the way swarms are engineered
by enabling few entities to control the outcome of large numbers of limited
robots. Applications include nanomedicine or the deployment of large scale
robotic systems for environmental monitoring.

In the scope of this project, we aim to control a swarm of 1000
kilobots~\cite{swarm} through the modification of their environment projected
as light pools on the kilobot arena. This will require an overhead camera to
observe the state the swarm and update the projected environment accordingly.
An algorithm will be designed to determine how the environment should change
for a given desired behaviour.

\bibliographystyle{ieeetr} \bibliography{bristol}

\end{document}
